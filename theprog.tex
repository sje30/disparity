%\documentstyle[harvard,egg,a4]{article}  % This is a hack for Emacs
%C-c C-r. It needs to find \documentstyle at the top of the doc.
% Title:  theprog
%
% Doc: /rsuna/home2/stephene/disparity/theprog.tex
% Original Author:     Stephen Eglen <stephene@cogs.susx.ac.uk>
% Created:             Mon Dec 11 1995
%

\documentclass[a4paper]{article}
\usepackage{alltt,theapa,pstimes,amstex,moreverb,psfig,mathptm}

\newcommand{\zbar}{\bar{z}}
\newcommand{\ztilde}{\tilde{z}}
\newcommand{\lkernel}{\bar{\Phi}}
\newcommand{\skernel}{\tilde{\Phi}}
\newcommand{\dskernel}{\tilde{\Phi'}}
\newcommand{\dlkernel}{\bar{\Phi'}}

\newcommand{\pdiff}[2]{ \frac{\partial #1}{\partial #2}}

\begin{document}

\title{ Spatial Disparity Program}
\author{Stephen Eglen}
\date{\today}
\maketitle

\begin{abstract}
  These are just a few notes on how to run and extend the spatial
  disparity model.
\end{abstract}

\section{Conventions}

JVS All file names are in {\bf bold}. All C procedure and variable
names are in {\em italics}. All C arrays are indexed from 0 to $n-1$,
where $n$ is the number of array elements.

\section{Running the Program}

To run the program, you need the program \texttt{testnet} in your path.  To
start the program type:

\texttt{testnet disp.prm}

where disp.prm is the parameter file that controls the behaviour of
the program.  JVS This file specifies a network with a $5\times1$
input array, and 1000 inputs. As the program runs, it will print up
the progress of Conjugate gradient descent (CG).

Two files are actually needed to run the program which are now described.

\subsection{Parameter Files}

The parameter file stores all the relevant parameters of the program
that can be changed.  Each parameter corresponds to a global variable
within the program, and so when modifying the program, the parameters
should be accessible everywhere.  Each parameter has a default value,
which can be changed by the parameter file.  (Otherwise the parameter
keeps its default value.)

To change a value of a parameter, simply write the parameter name and
its new value.  The amount of whitespace should not matter, and
neither should the order in which the parameters are listed.  Placing
a \# indicates that the rest of a line is a comment.  C style comments
can also be used in the file.

A list of all the parameters can be found in the file dispvars.h,
along with documentation.   The default values are set in the
procedure {\em setParamDefaults()} in the file {\bf testnet.c}.  An example
parameter file is given in Appendix \ref{egparam}.
 
Any text (eg parameter names misspellt)  not recognised by the parser
is simply copied to stdout.  If you have misspellt a parameter name,
it is worth checking the output as it comes up on the screen to see if
the parameter has the value you expect. (look for the text \texttt{*** System
Parameters ***} in the output )

\subsection{Network Files}
A second file is needed to tell the program how the network is
arranged.  This is called the network file.  An example network file
is given in Appendix \ref{egnet}

Comments are allowed in this file by using a \# only at the start of
the line.  The order of entries in the network file is important, and should be
as follows:

\subsubsection{Number of layers}

\texttt{nLayers $n$}

where $n$ is the number of layers in the network. For a network with one
input layer, one hidden layer and one output layer, $n=3$.  The number
of layers can be any value greater than 1.  (A value of two will just
create one input layer and one output layer.)

\subsubsection{Number of cells in each layer}

For each layer $0,1 \ldots n-1$, you need to specify how many cells
there are:

\texttt{layer i x y}

means that layer $i$ has $x \times y$ cells, arranged in a grid of x rows by
y columns.  Note that this number of cells does not include the bias cells.

\subsubsection{Layer information}

For each layer, you need to specify what kind of cells are in the
layer, and whether they provide a bias for the next layer.  This needs
to be done for layers $0..n-1$ in order.

\texttt{layer i actfn bias?}

\textbf{Activation function}: actfn = {LINEAR or TANH} to indicate the kind of
activation function each unit in this layer has. LINEAR is a bad
choice of name, and maybe it should be renamed to IDENTITY at some
point in time?


\textbf{Bias}: bias? = {bias or nobias} to indicate whether a bias
cell should be put into this layer.  This bias cell will project to
all cells in the next layer if it is present.  When telling the
program the number of cells in a layer you do not need to include the
bias cell. For example, if you have 5x2 cells in layer 0 and there is
also a bias cell in layer 0, then just say \texttt{layer 0 5 2} as
normal.  The activation and output of the bias cell is always set to
1.0.

Note that this program assumes that the output layer only has one
unit, and that this unit's output is copied into an array of virtual
output units, to produce the effect of shared weights across an array
of output units.

\subsubsection{Connectivity}
This part of the network file says how cells in the source layer are
connected to cells in the destination layer.  Cells in layer $i$ can
only receive input from cells in layer $i-1$.

Connectivity needs to be specified for each layer $1..n-1$, in order.  
There are two types of connectivity: full and local.

Full connectivity: To specify  full connectivity to layer $i$ from
layer $i-1$, you simply put:

\begin{verbatim}
connections to layer i
full
\end{verbatim}

Local connectivity:

To specify local connectivity, you need to say for each cell in the
destination layer, the rectangle of units in the source layer that it
receives input from.

For example, if layer 2 has two cells, and layer 1 has fifteen cells
arranged in 5 columns and 3 rows, then the following:
\begin{verbatim}
connections to layer 2
0 0 3 3
2 3 5 3
\end{verbatim}
says that the first cell in layer 2 receives input from layer 1 cells
in the region [0,0 to 3,3] and the second cell in layer 2  receives
input from layer 1 cells in the region [2,3 to 5,3].

\subsection{Training the network}

Once you have a parameter and network file, the program can then be
run.  (The parameter {\it checker} allows you to switch between CG and
the checker module.)

\subsection{Files produced during training}
As the program proceeds, several files are created.  

\subsubsection{Correlation}
After every line search, the correlation is tested, and written to the
file {\it corrn.dat}, which can be plotted. The correlation is also
written on the screen.

\subsubsection{Merit function}

The merit function is written every line search to the file
\textit{dispoutputs}, so that the merit function can be checked.

\subsubsection{z, zbar and ztilde}

z, zbar and ztilde are periodically written at a  line search $i$
(normally every 10 line searches) to
the files \textit{z.i, zbar.i, ztilde.i}.  

Depending on the topology of the output, these files can either be
viewed as a graph (1D) or as an image (2D).  In both cases however,
these files are raw text.

To view the file, eg \textit{z.50}, as a graph, I normally do the
following in gnuplot
\begin{verbatim}
gnuplot> plot "<sw z.50"
\end{verbatim}
where \textit{sw} is an auxiliary program.

If the file is to be viewed as an image, then I convert it to an image with
the following
\begin{verbatim}
snpgm z.50
\end{verbatim}
where $snpgm$ is another program.  This will show the weights as a PGM
file within xv.

\subsubsection{Weights files}

The weights are also periodically written out to the file $wts.i$,
where $i$ is the line search number.  This is mainly so that they can
then be used for later testing of the network.  The format of these
files is one weight per line in ASCII format.

\subsubsection*{Interpreting the weights files}

A simple script, \texttt{ew}, has been written to extract weights from
a weight file and put them in a suitable raw format such that they can
be converted to an image by \texttt{snpgm}.  
For example, to extract weights 0 to 8 (the first 9 lines of the
weights file wts.1000, since weights start from 0), and convert them into a 3
by 3 image, we do:

\texttt{ew 0 8 3 wts.1000 hidden0}

This will create the raw file hidden0, containing 3x3 weights, and an
image \texttt{hidden0.pgm} which can be viewed.

The general usage is: {ew $first$ $last$ $wid$ $wts$ $opfile$}, where

$first$ is the number of the first weight to extract.  $last$ is the
number of the last weight to extract.  $wid$ is the width of the image
to create. (The image is filled up on a row by row basis.)  $wts$ is
the name of the weights file to extract the weights from.  $opfile$ is
the name of the file to write the weights into.  Another file
opfile.pgm will be created with the image in it.

The image is created by filling in pixels on a row by row basis.  If
there are not enough weights to complete the last row, padding pixels
(set to the maximum weight found in the image) are included so that
the image can be made.

See Appendix \ref{egew} for an example of how to extract weights.

\subsubsection{Mask files}

To check that the convolution masks  $\lkernel, \skernel, \dlkernel,
\dskernel$ have been created.  These masks can
be viewed with either gnuplot or xv, in the same way as the other
output files.

The masks are assumed here to be exponential, with a cut of at 4 half
lives.  The size of the masks for the short and long range means can
be given in terms of $\lambda$ or in terms of the half life $t_h$. The
form of the exponential used here is:

\begin{align*}
  A & = A_0 e ^{- \lambda t} \\ 
  \intertext{When at the half life, $t_h$, $A_h = \frac{1}{2}A_0$:}
  \frac{1}{2}A_0 & = A_0 e ^{-\lambda t_h} \\
  \ln \frac{1}{2}& = - \lambda t_h \\
  t_h & =  \frac{\ln 2}{\lambda}
\end{align*}

\subsection{Testing the network}

At the end of training, you are left with a set of weights and other
output files.  These weights can then be fed back into the program to
test the program on another set of images that weren't seen during
training.

To do this, you need to create another parameter file which tells it
which images you want.  (The network file will stay the same.)  An
example parameter file is given in Appendix \ref{egcheck}.

To switch off learning, set $doLearning=0$, and set $results$ to the
name of the file where you wish the outputs of the network to be
written. Set $initWts$ to the name of the weights file you wish to
load into the network to test on the new images.

The program is then run, with this new parameter file.  The outputs of
the network will be written to the results file, and the correlation
between the outputs and the shifts array will be printed on the
screen.

(Relevant function:   void {\em checkNetPerformance()} in {\bf testnet.c})

\subsection{Sampling the input images}

$totalInputWid$ and $totalInputHt$ describe how big the two input
images and the shift file are.  Input vectors for the network are
created by cutting the input images (and the shift image) into boxes
of size $(inputWid + inputSkipX) \times (inputHt \times inputSkipY) $
For each box, an input vector of size $inputWid \times inputHt$ is
extracted from the top left hand corner of this box.  Similarly, the
appropriate shift value is taken from the centre of each box over the
shifts image.

As a diagnostic, the input vectors created are written to the file
\textit{inputs.test}, one per line of the file.  The corresponding
disparity values from the shifts image are stored in
\texttt{shifts.test}.

\subsubsection{Input Image Format}

The input image format is slightly different to the images output by
Jim's program that generates the images.  The basic difference is that
the first two lines of the image are deleted (this is the image size
and a blank line), and a trailing newline is added.    (This was done
so that the image format is fairly raw: if the image is x columns and y
rows, then the file is y lines long, with x floating point numbers on
each line.  Once in the raw format, I can then view the file as an
image with \texttt{snpgm}.)

The function that reads in images is void readInputFile(char
*inputFile, Array arr).

To save you editing the files by hand, a simple script
\texttt{conjimim} will convert the image into the proper format.  To
use this, simply do \texttt{conjimim image1.txt > im1.txt}, and
im1.txt will be in the proper format.

All of the data files have been kept in the directory
\texttt{/rsunq/vision/stephene/data/JIMS/}.

\subsubsection{Input Layer Size}

If the input images are being chopped up into vectors of size
$inputWid \times inputHt$, then because we have two input vectors (one from
each image) entering the network, the network file should have the line
\texttt{layer 0 x y}, where $x= inputWid$ and $y = 2 \times inputHt$.

\subsection{Size of virtual output layer}

The topology of the output layer is defined by the parameters
$outputWid$ and $outputHt$.  If $outputHt==1$, it is assumed that the
output units are arranged in a one dimensional chain, otherwise the
units will be in a 2d grid of size $outputWid \times outputHt$.

The number of units in the output layer must not be less than the
number of input vectors ($numInputVectors$), and ideally should be the
same (i.e $outputWid \times outputHt == numInputVectors$).  If there
are more output units than input vectors, the remaining output units
will have an output of 0.

Also, the following should hold:
\begin{align*}
outputWid = & \frac{totalInputWid}{inputWid + inputSkipX} \\
outputHt = & \frac{totalInputHt}{inputHt + inputSkipY}
\end{align*}


\subsection{Run time}

For most of the 1d examples, the program will take just a couple of
minutes to run on a workstation. The 2d examples will take much longer
however.    In the largest experiment set up, (gauss600c.prm) using a
100x100 output array, it took around a day to do 500 epochs on a fast
workstation!  However, the smaller 2d examples (eg gauss600.prm - a 20
by 20 output layer) also run quite quickly.



\section{Updating the program}

All of the source files are stored in  the directory
\~{}stephene/disparity/.  

\subsection{General Program Description}

\subsubsection{Initialisation}

First of all, the parameter file is read in to set the parameters of
the network.  Then the network file is read in, and all of the
activation data structures are created, along with the weights.  The
input images are chopped into input vectors.

\subsubsection{Conjugate Gradient}

Conjugate gradient does most of the hard work, calling the evaluation
function and then creating the derivative vectors.  When the
evaluation function is worked out, it will also create the derivative
vectors, and therefore some work is done but not taken advantage of
here.  

For each call to determine the evaluation function, all of the input
vectors are presented to the network, one at a time.  The output of
the network is then stored in the virtual array $z$ of output units.
(The output of the network to the ith input vector is stored in z(i).)
After all of the inputs have been presented, $z$ is then convolved
with the two masks to create $\zbar$ and $\ztilde$.  From there,
the errors for the top layer of virtual cells $z$ are created, and
then propagated backwards to the hidden layer cells.  These errors are
then passed back to conjugate gradient.

Before using CG, it is worthwhile to use the gradient checker, by
setting the $checker$ parameter to 1, and watching to see if the ratio
calculated comes to 1.000.

\subsection{Relevant functions}

This section gives the names of the major functions that will be
useful to follow when extending the program.

{\em netmain()} is the main function of the program and calls the
initialisation routines, and then either tests the network, or trains
it using conjugate gradient.

\subsubsection{Initialisation}

Initialisation is done by the function {\em setUpNetwork()}.  This reads in
the parameter file ({\em getParams()}) and then creates the network data
structures ({\em readnet()}).  Weights are initialised to random values
({\em initWtsRnd()}).  The input vectors and actual disparity values are read
by {\em createInputVectorsAndShifts()}.  

{\em readnet()} is quite a large function which does several things.
Firstly, it allocates space for the weights ({\em createWeights()}). It
allocates space for the activations array actnInfo (which stores
current network activity) and the structure {\em allActns} (which stores all
of the activations for one iteration - i.e for each input vector).

The main task for this function is to then read in the network file,
and decide how many weights will be needed to set the network up, and
so forth.  This is done on a layer by layer basis.  For each cell in
each layer, we find out from the network file which cells it is taking
input for, allocating weights contiguously. (So if a cell in layer 1
receives inputs from 5 layer 0 cells, then the 5 weights are
contiguous in the weight vector.)  This is done using {\em connectCells()}.
At the same time, the preCellInfo and cellInfo structures are
updated. (The preCellInfo structure says which cells in layer $i+1$
are connected to by a cell in layer $i$. Conversely, cellInfo says
which cells in layer $i$ provide input to a cell in layer $i+1$.)

The global arrays for $z, \zbar, \ztilde$ and so on are created using
{\em createZs()}.  Similarly, the error vector arrays dw and onedw are
created by {\em createdw()}.  

The final job of initialisation is to create either 1D or 2D masks for
calculating U and V.  (Using either {\em createMasks()}  or {\em createMasks2()}. )

\subsubsection{Testing Network Performance}

If no learning is to take place, then the routine {\em checkNetPerformace()}
is used to see how well the net performs on data that it hasn`t seen
before.

To do this, it reads in results from the initWts parameter using
{\em readWts()}.  Each input vector is then inserted into the activations
array (getNextInputVector), and the activation of the output cells for
that input calculated.  All of the activations are then stored into
the {\em allActns} structure by {\em storeActivations()}.

Once all of the inputs have been presented, the correlation is
determined by {\em Rvec\_correlate()} and the $z$ array output to the results
file using {\em writeArray()}. After this, the program exits, as no learning
is to take place.

\subsubsection{Conjugate Gradient and Checker Routine}

If learning is to take place, then the conjugate gradient routine
(cg\_williams) is called.  Importantly, the merit function is called
{\em evalFn()} and the function returning the weight change vector is called
{\em evalPartials()}.  

The program normally runs for a fixed number of line searches (set by
the maxiterations parameter), but also will print the correlation
between output and image disparity.  This is done by {\em finishedFn()}.

Both {\em evalFn()} and {\em evalPartials()} are simple wrappers around the
function {\em calcMeritAndPartials()}, which does most of the hard work,
evaluating the function and then creating the weight change vector.


\subsubsection*{Network Activation}

The function {\em clearActivationArray()} sets all of the units activations
and outputs to 0.0.  {\em setBiases()} then sets the activation and output
of the cells to 1.0.  The activation and outputs of the inputs is then
set by copying across the relevant input vector.  

Once the input cells have an activation and input, the activation of
all layers is calculated by {\em calcAllActivations()}.

For this learning rule, we normally need to do batch learning.  So,
all of the activations and outputs of each cell needs to be remembered
for later use for calculating the error vector and creating the array
of virtual outputs.  {\em storeActivations()} does this job of copying the
cell activations into the {\em allActns} structure.

\subsection{Global Variables}

Several global variables have been used in the program.  All of the
system parameters are global variables, and are described in the file
{\bf dispvars.h}.  Not all of the global variables however are
parameters (i.e. changeable by the parameter file) -- these
variablesare documented in the file {\bf dispglobals.h}.

JVS Variables listed in *.prm file can be changed directly by user to
alter values in {\bf dispvars.h}.  In contrast, variables in {\bf
dispglobals.h} are changed by program as a consequence of user
altering values in *.prm file.
 
Finally, four global variables -- \texttt{weightInfo, netInfo,actInfo,
layerInfo} are declared in {\bf dispnet.h}.  They are declared here,
rather than in {\bf dispglobals.h} since these variables are of complex
types - these types also being defined and documented in this file.

\subsection{Main Data structures}

Most of the data structures discussed here are defined in the file 
{\bf dispnet.h}.

\subsubsection{Array}

The array data type stores a 2d  array as  a 1 dimensional vector of
length $wid \times ht$.  By convention if the array is 1D, then $ht = 1$.

JVS The {\em data} slot in {\em Array} is a pointer to the first element of an
array with $ht \times wid$ elements.

\subsubsection{layerInfo}

For each layer $i$ of the network, layerInfo[i] is a structure
containing the details of each layer: the number of cells, their
activation type, and whether this layer also provides a bias input to
the next layer.  

For each cell $j$ in a layer, preCellInfo[j], a component of the
layerInfo stores the details about the weights and cells in layer
$i+1$ that this cell connects to.  Hence, by looking at this
structure, it will allow you to work out the error for this cell,
given the errors for cells in layer $i+1$.

\subsubsection{cellInfo}

For cell $j$ (numbered relative to the number of cells in this layer)
in layer $i$, layerInfo[i].cellInfo[j] shows which cells in layer
$i-1$ provide input to cell $j$.  This structure is used by the
function {\em calcActivation()} to calculate the activation and output of
the cell given some input.

\subsubsection{weightInfo}

WeightInfo stores the actual weights of the network as a 1d vector in
data, starting from 0.   preCell[i] tells you the number of the
presynaptic cell that this weight is used for. postCell[i] tells you
the number of the postsynaptic cell that this weight is used for.

\subsubsection{Bias cells}

If cells in layer $i$ need bias input as well as input from cells in
layer $i-1$, then an extra cell is placed in layer $i-1$.

\subsubsection{activationInfo}

Each cell in the network is given a unique number, starting from 0.
These numbers are allocated contiguously from 0.  For example, given
the following network file:

\begin{verbatim}
# Simple 3 layer network
nLayers 3
layer 0 5 2
layer 1 5 2
layer 2 1 1

layer 0 LINEAR bias
layer 1 TANH nobias
layer 2 LINEAR nobias
connections to layer 1
full
connections to layer 2
full
\end{verbatim}

These are the activation cell numbers (printed by {\em printNet()} and
written to the file \texttt{actn.info} when the program has set up the network ):  

\begin{verbatim}
** Activations **

Layer 0
 0  1  2  3  4 
 5  6  7  8  9 
Bias 10

Layer 1
11 12 13 14 15 
16 17 18 19 20 

Layer 2
21 
\end{verbatim}

The $activationInfo$ structure stores the internal activations $actn[i]$ and
outputs $op[i]$ of each cell $i$.  In the above example, units 0..9
will store the current input vector and unit 10 is the bias unit that
projects to cells in layer 1.  (These unit numbers are the unit
numbers referred to in preCellInfo and cellInfo).  actn[21] will
store the activation of the output unit, and op[21] stores the output
of the output cell.

In this context, the activation of a cell is the dot product of the
weights and the inputs. The output of the cell is the result of
passing the activation through the activation function (normally
either linear or tanh).

\subsubsection{allActns}

JVS This is the main structure used to store the inputs, outputs
and error terms of every cell for all input vectors. The $allActns$ structure
contains 5 slots. Each slot is a 2D array of dimensionality
$dim[numInputVectors][numUnits]$. The 5 slots are: 

$num$ The length of this array.  $i$ subscript can range from 0 to
$i-1$. Normally, $i$ is equal to  $numInputVectors$ .
 
$allActs[i]$ Array of total input to each of $numUnits$ units in network.

$allOps[i]$ Output value of each of $numUnits$ units in network.

$errors[i]$: Array of error values, one per unit in network.

$numUnits$: Number of units in network.

If there are $x$ input vectors, then each input vector is loaded into
the input units of the activations array (by  {\em getNextInputVector(int
vecnum)}) and then the activation propagated through all layers to the
output cell. (This is done by the procedure void {\em calcAllActivations()}.)

Once the activations, errors and outputs have been calculated for a given
input vector, all of the activations and outputs of the net need to be
stored for later use by the functions to calculate the merit function
and error vector.

Hence, for input vector $x$, $allActs[x]$ is a pointer to a vector of
activation values -- these activation values are the values of all the
cells in the network given the $x$th input vector.  $allOps[x]$ stores
the corresponding output values of the cells given that input.
Finally, $errors[x]$ stores the vector of error measures given the $x$th
input vector.

JVS Once all input vectors have been processed by the net, the 
output of each input vector is stored by $getZ()$ in $z.data$.

\subsubsection{Calculating error vectors}

The main function to present all the inputs and calculate the outputs
and error vectors is  {\em calcMeritAndPartials()} in {\bf testnet.c}.

Once all of the input vectors have been input to the network, and the
array of output values $z$ have been created, $\ztilde$ and $\zbar$ arrays
are created by convolving the output z with the short and long range
masks (masks are created using the function void {\em createMasks()} in
{\bf dispmasks.c})

$U$ and $V$ are then calculated by subtracting either $\ztilde$ or
$\zbar$ from $z$.  The error measures $\pdiff{U}{X}, \pdiff{V}{X}$ are
also created by convolution - see Appendix \ref{maths.app} for
details.  These values then provide error measures for the top layers
of cells, which are then stored in the {\em allActns} data structure
using the routine {\em storeTopLayerErrors(da)}, where $da$ is the
array of error vectors for the output cells only.  These error values
are then propagated back to earlier layers by the back prop method,
implemented in {\em propagateErrors()}.

Finally, once the error values $\delta_a$ have been computed for all
cells in the network, the actual weight changes for each unit need to
be calculated using $\Delta w_{uv} = \delta_v z_u$, where $\delta_v$
is the error on cell $v$, $z_u$ is the output of cell u and $w_{uv}$ is
the weight connecting cell $u$ to $v$.  If there are $x$ input
vectors, then we will get $x$ weight change vectors, which are simply
added up to produce the final weight change vector, which is stored in
the global array $dw$.  This is done by the function {\em createPartials()}.

\subsubsection{Testing the network}

When it comes to testing the network on new images, the program simply
loads in some weights previously created during a training session,
reads in the input vectors, and creates the array of virtual outputs
$z$, which are then saved to the results file.  The correlation
between these outputs and the actual disparity outputs is then
displayed on the screen.

\subsection{Overview of how the weight sharing  works}

To create an array of virtual outputs $z$ of size $outputWid \times
outputHt$, there should be $outputWid \times outputHt$ input vectors.
The $i$th input vector is presented to the network and the output of the
output cell calculated.  This output is then stored in $z(i)$.  This
assumes that the network has only one cell in the output layer.

If you want to have $y$ cells in the output layer, then you will need
to make several changes:
\begin{enumerate}

\item Change the size of $z$ so that it is now an array of size $y
  \times numInputVectors$ (normally $numInputVectors == outputWid
  \times outputHt$).  This is done by the function {\em createZs()} in
  {\bf dispnet.c}.

\item {\em Change getZ()} in {\bf dispnet.c}.  This function will extract the
  output of the network for each input vector and store them in the $z$
  array.  So if you want to change the number of values that are
  extracted from the network and put into the $z$ array, do it here.

\end{enumerate}


\subsection{Source Files}

Here is a list of all of the relevant source files and a brief
description of their contents:

\begin{description}

\item[bp\_check\_deriv.c]
Checker routine to see that the derivative vector is being computed ok.

\item[cg\_williams\_module.c and .h] 
The conjugate gradient engine, adapted from Jim Stone.

\item[convolve.c] 

The 1d and 2d convolution routines (both assuming wrap around in the
masks and input images).

\item[dispcorrn.c] Correlation routines.

\item [disperrors.c] 
  Calculate the errors for the output cells and propagate them
  backwards to previous layers.  

\item[dispinputs.c] 
Read in the input images and chop them up into input vectors to be
presented to the network.

\item[dispmasks.c] 
Create the masks to be used for the convolutions.

\item[dispnet.c] 
Calculates the activations of units in the network given some input.

  \item[dispscan.c]
    This file is generated automatically from the lex file dispscan.l.
    The lex file stores the instructions on how to parse the parameter file.

\item[dispvars.h] 
Header file that stores the definition of all of the parameters that
can be changed by the parameter file.

\item[dispwts.c] 
Routines for handling the weights data structures.

\item[readnet.c] 
  Read in the network file and create the network data structures.

\item[testconvolve.c] 
Test the convolution (1D and 2D) routines.

\item [testnet.c and .h]
  Contains all the high level functions to run the program.  Most of
  the hard work is done in the other files!


\end{description}



\subsection{The makefile}

To recompile the program, type \texttt{make testnet}.  The makefile
also has rules to convert the lex file for scanning in the parameter
file into the c file.  This should all be automatic.

\subsection{New parameters}

To help add new parameters, run the script \texttt{newparam}.  This
will provide you with the relevant text to put in the files
\texttt{dispvars.h, dispscan.l, testnet.c}.   For example:

\texttt{newparam int inputSkipY}


\subsection{Masks}

At the moment, only exponential functions have been used as the masks
(1d and 2d) for producing $\ztilde$ and $\zbar$ from $z$.  To use
other functions,  you will need to modify the function {\em createMasks()}.
The masks for the derivatives  ($\dlkernel, \dskernel$) are calculated
from the original masks ($\lkernel, \skernel$), so you will only need
to change the routines to create uMask and vMask.  To make new masks,
it is probably easiest to take a copy of the function
{\em create2dExpMask()}.


\subsection{TAGS}

To help navigate around the C source functions, a TAGS file has been
created, which is useful for finding functions in emacs.   When the
cursor is located on top of a function name, you should be able to
type \texttt{M-.} and then return a couple of times to get the
definition of that function.

The tags finding function will also find macro definitions and typedef
statements, but not global variables unfortunately.  

If new functions are added to the existing files, you will need to do
\texttt{make TAGS} to update the tags file.  If you add new files to
the program, you will need to add this file to the TAGS dependency
list in the makefile, and then remake the tags file.


\subsection{Other Libraries }

The program should not require any other libraries of mine to run.


\subsection{How to not weight share}

At the moment, weight sharing is implemented, so that the same weights
are used for each input vector.  

If we are not going to weight share, then we need to create one big
weight array storing all of the weights.  This can be in a separate
array of size $numInputVectors \times nw$, where $nw$ is the number of
weights in the normal network (ie. weightInfo.numWts).  Let us call
this new array $allwts$ for example.

This array can be created once no more weights are required for the
normal network (for example, after the {\em noMoreWeights()} function
in {\em readNet()}).  These weights will then need to be given random
initial weights - currently this is done for the normal network by
\emph{initWtsRnd()} in the file \textbf{testnet.c}.

Once the weights have been allocated, the appropriate weights from
$allwts$ need to be copied across to the weightInfo.data structure
when the network activation is being calculated.  So for example, in
the function \emph{calcMeritAndPartials()}, we will need to do
something like \emph{copyWeights(vecnum)} before we call
\emph{calcAllActivations()} for each input vector.  

Next, we will also need to copy the weights across into
$weightInfo.data$ when we are working out the weight change for
each weight, in the function \emph{calcErrors()}.  This will need to
be done before the line \emph{wts = preCellInfo[unit].wts} in this
function.  To be clever, the
call to \emph{copyWeights()} should not be in the \emph{calcErrors}
routine, as this is called possibly several times per input vector.
So, it is probably better to put the call to \emph{copyWeights()} in
the function \emph{propagateErrors()}. 

Other changes will also be needed - but these should be much
simpler. For example, when calling conjugate gradient or the checker
routine, we will needed to pass the new weight array $allwts.data$
rather than $weightInfo.data$.  

Also, the weight I/O functions will need to be changed so that when we
output the weights (mainly within the \emph{cg\_williams} routine),
then we output $allwts$ rather than $weightInfo$.



\subsubsection{Testing the net}

Testing the network on inputs unseen during training is done by
\emph{checkNetPerformance()}.  This will need to be modified in a
similar fashion to \emph{calcMeritAndPartials()}, so that when each
new input is presented, the corresponding weights are also loaded in.


\section{Results}

Some example runs are given in Appendix \ref{egruns}.  A summary of
all of the runs performed with this net is given in the file
\texttt{~/disparity/ResultsSummary}.  All of the results are currently
stored in the directory \texttt{/rsunq/vision/stephene/dispruns},
although most of them have been compressed to save space.


\section{To do}

\begin{enumerate}
\item 
  At the moment, the error propagation routine {\em calcErrors()} in
  {\bf disperrors.c} assumes that the layer has tanh activation function.
  This needs to be tested for, rather than assuming that we should take
  the derivative of {\em tanh()} for the cells. - {\bf Done Fri Dec 15
    1995}

\item 
  For each input vector, the program needs to keep a copy of each units
  activation, output and error.  For a large number of input vectors (eg
  40,000 for a 200x200 network), this causes the program to come to a
  halt, complaining about not being able to allocate enough data. 


\item Both evaluation calls within CG -- to evaluate the merit
  function and to provide the error vector -- evaluate both the merit
  function and the error vector for simplicity.  However, it would be
  more efficient only to evaluate the functions if the weights have
  changed since the last call to these functions.

\item Bias cells.  If a bias cell is included in a layer, then its
  activation and output are set to 1.0 by the routine {\em setBiases()}.
  However, this probably assumes that the bias cell has a linear
  activation function.  If a bias cell is included in the hidden layer
  at the moment (ie. the output cell receives a bias input), then this
  cell will be treated as a tanh unit, when in fact it is a linear
  unit.   This problem does not occur at the moment because there is
  no bias unit included in the hidden layer.  To solve this, I guess
  {\em setBiases()} will need to check what activation function is used for
  the current layer, or calcErrors will need to be modified.
  \textbf{Done - Fri Dec 15 1995}

\item Input cells.  The function {\em getNextInputVector()} also currently
  assumes that the activation function of the input cells is the
  identity function.  \textbf{Fri Dec 15 1995 - Normally the input
    cells activation function will be the identity function, and so
    this function just prints an error and exits if the identity
    function is not used for input cells.}


\end{enumerate}

\section{Other programs}

The following auxiliary programs will be useful. (I use these quite
often.). All of them are scripts and can be found in
\texttt{~/disparity/dispbin}.  One of the sub programs,
\texttt{2drawtopgm} is a binary. If it needs recompiling, it can be
remade in the \texttt{src/} subdirectory.

\begin{description}
\item[sw] Strip Whitespace from files and throw out numbers one line
  at a time.

\item[snpgm] Show the Newest file as a PGM file. Converts raw data
  files into a PGM image, and then starts up xv.  Will create a pgm
  file as a side effect.

\item[newparam] Generate all the relevant text to edit into the source
  files for creating a new parameter.


\item[conjimim] Convert Jims image files into a raw format.

\item[get2dsize] Tells you  the dimension of raw image files (number
  of columns by number of rows).


\item[ew] Extract weights from the weights file.  This is mainly
  useful for seeing what the weights going into a hidden / output unit
  look like.

\end{description}

\clearpage
\appendix

\section{Example Parameter File} \label{egparam}

\subsection*{Filename: gauss600.40x40.prm}

\verbatiminput{gauss600.40x40.prm}
\clearpage


\section{Example Network File} \label{egnet}

\subsection*{Filename: gauss600.net}
\verbatiminput{gauss600.net}

\clearpage
\section{Example parameter file for testing the net after training}
\label{egcheck} 

\subsection*{Filename: checkegg.prm}
\verbatiminput{checkegg.prm}


\section{Merit Function and Learning Rule}
\label{maths.app}

\subsection{Notation}

\begin{center}
  

\begin{tabular}{cp{8cm}}
  Term & Meaning \\ \hline
  $x$ & total input to a cell \\
  $z$ & Output of a cell \\
  $\zbar$ & Long range mean of $z$ \\
  $\ztilde$ & Short range mean of $z$ \\
  $i$ & Subscript for input cells \\
  $j$ & Subscript for hidden cells \\
  $w_{ij}$ & Weight from  input cell $i$ to hidden layer cell $j$\\
  $w_{jk}$ & Weight from  hidden layer cell $j$ to output layer cell $k$\\
  $a,b,k$ & Subscript for output cells \\
  $\skernel$ & Short range kernel for convolution to form $\ztilde$\\
  $\lkernel$ & Long range kernel for convolution to form $\zbar$\\

\end{tabular}
\end{center}

The merit function $F$ is defined as:
\begin{align*}
  F &= \log \frac{V}{U} \\ \intertext{where $V$ is the long range
    variance and $U$ is the short range variance:} U & = \frac{1}{2}
  \sum_k ( \ztilde_k - z_k ) ^2 \\ V & = \frac{1}{2} \sum_k ( \zbar_k
  - z_k ) ^2 \\ \\ \intertext{$\zbar_k$ is the long range mean, and
    $\ztilde_k$ is the short range mean for an output cell $k$,
    defined as:} \zbar_k &= \sum_a \lkernel_{a-k} z_a \\ \ztilde_k &=
  \sum_a \skernel_{a-k} z_a
\end{align*}
where $a$ ranges over output cells.

To calculate the derivative of the merit function with respect to each
weight, we need to determine:
\begin{align*}
\pdiff{F}{w} &= \frac{1}{V} \pdiff{V}{w} - \frac{1}{U} \pdiff{U}{w}
\end{align*}

Hence, after all of the input has been presented to the network, we
calculate $U$ and $V$.  We then need to calculate $\pdiff{U}{W}$, or
to use the back propagation approach, we need to calculate
$\pdiff{U}{x_a}$.




\subsection{Deriving the Error Derivatives}

In a similar fashion to the Back Propagation algorithm, once the
errors $\delta$ are known for the output layer, they can then be used
to calculate weight changes between output and hidden layer cells. The
errors can also then be propagated back to the hidden layer cells so
that the weights between input and hidden layer cells can be changed:

Given $\delta_k$ for output layer cells, we can then calculate:
\begin{align*}
\Delta w_{jk} & = \delta_k z_j \\
\intertext{We can then calculate $\delta_j$ from the $\delta_k$:}
\delta_j &= g'(x_j) \sum_{k} w_{jk} \delta_k
\end{align*}
where $g'(x)$ is the derivative of the activation function for the
hidden layer cells.



\subsubsection{Computing $\pdiff{U}{x_a}$}


From the normal back propagation maths, $\delta_k = \pdiff{E}{x_a}$,
and so we need to find $\pdiff{E}{x_a}$:
\begin{align}
  U &= \frac{1}{2} \sum_k (\zbar_k - z)^2 \\
  \pdiff{U}{x_a} &= \sum_k (\zbar_k - z) \left( \pdiff{\zbar_k}{x_a} - 
  \pdiff{z_k}{x_a} \right) \label{u_xa1}\\
\intertext{To find   $\pdiff{z_k}{x_a}$:}
z_k = f(x_k)\\
\intertext{Here the activation function of the output cells is the
  identity function, and so:}
\pdiff{z_k}{x_a} & = \left\{
  \begin{array}{ll}
    1 & \text{if $a=k$} \\
    0 & \text{otherwise} \\
  \end{array}
  \right. % This is an invisible delimiter.
  \label{dzk_dxa}\\
  \intertext{To find $\pdiff{\ztilde_k}{x_a}$:}
  \ztilde_k  &= \sum_b \skernel_{b-k} z_b \\
  \pdiff{\ztilde_k}{x_a} &= \sum_b \skernel_{b-k} \pdiff{z_b}{x_a} 
\label{zb_xa}\\
\intertext{Using Equation \ref{dzk_dxa}, we can then see that equation
  \ref{zb_xa} is zero unless $a=b$:}
  \pdiff{\ztilde_k}{x_a} &= \skernel_{a-k} \label{ztildek_xa}
\end{align}
The values for $\pdiff{z_k}{x_a}$ and $\pdiff{\ztilde_k}{x_a}$ from
equations \ref{dzk_dxa} and \ref{ztildek_xa} can then be substituted
back into equation \ref{u_xa1} to give:
\begin{align}
  \pdiff{U}{x_a} &= \left\{
    \begin{array}{ll}
      \sum_k (\ztilde_k - z) ( \skernel_{a-k} - 1) & \text{if $a=k$}
      \\
      \sum_k (\ztilde_k - z) ( \skernel_{a-k}) & \text{otherwise}\\
    \end{array}
    \right.\\
\intertext{So, if we simply define a new kernel $\dskernel$:}
\dskernel_a &= \left \{
  \begin{array}{ll}
    \skernel_a -1 & \text{if $a=0$} \\
    \skernel_a & \text{otherwise} \\
  \end{array}
  \right. \\
\pdiff{U}{x_a} &= \sum_k (\ztilde_k - z_k) \dskernel_{a-k} \\
\intertext{The $\delta_a$ can be computed as a convolution of the vector $(
\ztilde_a - z )$.  By symmetry, a similar expression can be derived
for $\pdiff{F}{x_a}$, so that the final $\delta_a$ is:}
\delta_a = \pdiff{F}{x_a} &= \frac{1}{V} \pdiff{V}{x_a} - 
\frac{1}{U} \pdiff{U}{x_a}
\end{align}


And once $\delta_a$ is defined for the output cells, we can then back
propagate the errors to the earlier layers, and then calculate the
weight change.  In general for a weight connecting cell $u$ to cell
$v$, the weight change for the weight $w_{uv}$ is $\Delta w_{uv} =
\delta_v z_u$.


\section{Example weight extraction}

\label{egew}
This section gives a brief example on how to extract the weight
vectors.

If the network is run on the parameter file
~/disparity/gauss600.40x40.2.prm,  then after training, the
\texttt{weight.info} file will say how the weights are broken down.
This network is set up as follows (taken from gauss600.5x5.net:)

\begin{verbatim}
...
layer 0 5 10
layer 1 1 3
layer 2 1 1

# Layer information

layer 0 LINEAR bias
layer 1 TANH nobias
layer 2 LINEAR nobias
...
\end{verbatim}
Hence, there are 51 ( $2 5 \times 5$ input vectors + 1 bias) cells
going into each of the three hidden units. Also, there are $3$ weights from
hidden to output cells.

The final weights (from \texttt{wts.1000} can be extracted as follows
into the files hidden.[012].pgm

\begin{verbatim}
ew 0 49 5 wts.1000 hidden0
ew 51 100 5 wts.1000 hidden1
ew 102 151 5 wts.1000 hidden2
ew 153 155 3 wts.1000 op0
\end{verbatim}

(Note that the bias cell is cell 50, and these are not printed, hence
the jump of one between end values of one set of weights and the start
of the next set of weights -- jumping over weights 50, 101 and 152.)


\section{Example runs}
\label{egruns}
This is the README file in the disparity directory. Afterwards are the
resulting plots and images.


\verbatiminput{README}

\clearpage
\subsection{1D Outputs}

\begin{figure}[h!]
\centerline{\psfig{figure=/rsunq/vision/stephene/dispruns/testthencheck/sin5x1000/sintrainz.ps,width=300pt}}
\centerline{\psfig{figure=/rsunq/vision/stephene/dispruns/testthencheck/sin5x1000/sintraincorrn.ps,width=300pt}}
\caption{Above: z, zbar and ztilde after training on the sin 5x1000
  data set. Below: Correlation and merit function.}
\label{sintrain.p}
\end{figure}

\clearpage

\subsection*{1D Outputs - Test Results}
\begin{figure}[h!]
  \centerline{\psfig{figure=/rsunq/vision/stephene/dispruns/testthencheck/sin5x1000/sintestz.ps,width=300pt}}
\caption{Output after testing sin5x1000 period 200 data on network
  learnt using sin period 1000 data.}
\label{sintest.p}
\end{figure}

\clearpage



\subsection{2D Outputs}


\begin{figure}[h!]
  \centerline{\psfig{figure=/rsunq/vision/stephene/dispruns/testthencheck/gauss600.40x40/z.1000.ps,width=300pt}}
  \vspace{0.3cm}
  \centerline{\psfig{figure=/rsunq/vision/stephene/dispruns/testthencheck/gauss600.40x40/gausscorrn.ps,width=300pt}}
  \caption{Above: The final output array $z$ after training on the gauss 600 x
    600 data (gauss600.40x40.prm).  Below: correlation and merit function.}
  \label{gausstrain.p}
\end{figure}

\clearpage

\subsection*{2D Outputs - Test Results}

\begin{figure}[h!]
\centerline{\psfig{figure=/rsunq/vision/stephene/dispruns/testthencheck/gauss600.40x40/eggresults.dat.ps,width=300pt}}
\caption{The image file eggresults.dat.pgm - showing the output of the net
  trained on the gaussian data when presented with the egg box data (checkegg.prm).}

\label{gausstest.p}
\end{figure}


\clearpage

\subsubsection{Weights}

These are the final weights as a result of learning:



Weights to hidden unit 0 min -1.283811 max 1.048920

\centerline{\psfig{figure=/rsunq/vision/stephene/dispruns/testthencheck/gauss600.40x40/hidden0.ps,width=200pt,angle=90}}


Weights to hidden unit 1 min -1.421594 max 1.689905

\centerline{\psfig{figure=/rsunq/vision/stephene/dispruns/testthencheck/gauss600.40x40/hidden1.ps,width=200pt,angle=90}}


Weights to hidden unit 2: min -0.413099 max 0.379919

\centerline{\psfig{figure=/rsunq/vision/stephene/dispruns/testthencheck/gauss600.40x40/hidden2.ps,width=200pt,angle=90}}


Weights to hidden unit 3: min -1.274979 max 1.064472

\centerline{\psfig{figure=/rsunq/vision/stephene/dispruns/testthencheck/gauss600.40x40/hidden3.ps,width=200pt,angle=90}}


Weights to hidden unit 4: min -1.985908 max 1.680910

\centerline{\psfig{figure=/rsunq/vision/stephene/dispruns/testthencheck/gauss600.40x40/hidden4.ps,width=200pt,angle=90}}

\clearpage

Weights to output  unit 0: min -0.470672 max 0.666059

\centerline{\psfig{figure=/rsunq/vision/stephene/dispruns/testthencheck/gauss600.40x40/op0.ps,width=200pt}}


% Start-Of-Trailer

%\bibliographystyle{theapa}
%\bibliography{gen}

\end{document}



